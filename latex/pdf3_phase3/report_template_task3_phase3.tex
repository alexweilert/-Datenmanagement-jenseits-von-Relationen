\documentclass[11pt]{scrartcl}

\usepackage[top=1.5cm]{geometry}
\usepackage{url}
\usepackage{float}
\usepackage{listings}
\usepackage{xcolor}
\usepackage{graphicx}
\usepackage{booktabs}

\setlength{\parindent}{0em}
\setlength{\parskip}{0.5em}

\newcommand{\youranswerhere}{[Your answer goes here \ldots]}
\renewcommand{\thesubsection}{\arabic{subsection}}

\lstdefinestyle{dmrsql}{
  language=SQL,
  basicstyle=\small\ttfamily,
  keywordstyle=\color{magenta!75!black},
  stringstyle=\color{green!50!black},
  showspaces=false,
  showstringspaces=false,
  commentstyle=\color{gray}}

\lstdefinestyle{dmrJava}{
  language=JAVA,
  basicstyle=\small\ttfamily,
  keywordstyle=\color{magenta!75!black},
  stringstyle=\color{green!50!black},
  showspaces=false,
  showstringspaces=false,
  commentstyle=\color{gray}}

\title{
  \textbf{\large Projektaufgabe 3 } \\
  Phase 3 – Implementierung des XPath Accelerators (3.5 P) \\
  {\large Datenmanagement jenseits von Relationen}
}

\author{
  Gruppen Nummer (e.g. A1, B5, B3) \\
  \large Lastname1 Firstname1, StudentID1 \\
  \large Lastname2 Firstname2, StudentID2 
}

\begin{document}

\maketitle\thispagestyle{empty}

Dieses Reporting Template dient der Vorbereitung der Abgabe von Phase 3. 

\subsection*{Verkleinern der Fensteranfrage (0.5 P)}

Zeigen Sie, dass die Berechnung der Achsen für die Beispiele aus Phase 1 hier die gleichen Ergebnisse erzeugen:

\begin{table}[h]
	\centering
		\begin{center}
			\begin{tabular}{ l | c c }
				\toprule
				Achse & Ergebnisknoten ID's & Ergebnisgröße\\
				\midrule
				ancestor & \{0, 1, 2, 3\} & 4 \\
				descendants & \{35, 36, 37, 38, \ldots, 62\} & 28 \\
				following SchmittKAMM23 & \emptyset & 0 \\
				preceeding SchmittKAMM23 & \{35\} & 1 \\
				following SchalerHS23 & \{48\} & 1 \\
				preceeding SchalerHS23 & \emptyset & 0 \\
				\bottomrule
			\end{tabular}
			\end{center}
	\caption{Ergebnisse der XPath-Berechnung auf Edge Modell aus Phase 1}
	\label{tab:ErgebnisseDerXPathBerechnug1}
\end{table}

\begin{table}[h]
	\centering
		\begin{center}
			\begin{tabular}{ l | c c }
				\toprule
				Achse & Ergebnisknoten ID's & Ergebnisgröße\\
				\midrule
				ancestor & \{0, 1, 2, 3\} & 4 \\
				descendants & \{35, 36, 37, 38, \ldots, 62\} & 28 \\
				following SchmittKAMM23 & \emptyset & 0 \\
				preceeding SchmittKAMM23 & \{35\} & 1 \\
				following SchalerHS23 & \{48\} & 1 \\
				preceeding SchalerHS23 & \emptyset & 0 \\
				\bottomrule
			\end{tabular}
			\end{center}
	\caption{Ergebnisse der XPath-Berechnung des XPath-Accelerators aus Phase 3 mit Verkleinerung des Fensters}
	\label{tab:ErgebnisseDerXPathBerechnug2}
\end{table}



\subsection*{Schemaerstellung (1 Punkt).}
Geben Sie die Create Table statements aller von Ihnen angelegten Tabellen an:
\begin{lstlisting}[style=dmrsql]
CREATE TABLE IF NOT EXISTS node (id int ,s_id TEXT, type TEXT,
			content TEXT)
CREATE TABLE IF NOT EXISTS edge (parents INT, childs INT)

CREATE TABLE IF NOT EXISTS accel (id INT PRIMARY KEY, post INT,
			s_id TEXT, parent INT, type TEXT)

CREATE TABLE IF NOT EXISTS content (id INT PRIMARY KEY, text TEXT)

CREATE TABLE IF NOT EXISTS attribute (id INT PRIMARY KEY, text TEXT)

CREATE TABLE IF NOT EXISTS height (id INT PRIMARY KEY, height INT)
\end{lstlisting}

\subsection*{Zugriff mit nur einer Achse (1 Punkt).}
Zeigen Sie, dass die Berechnung der Achsen für die Beispiele aus Phase 1 hier die gleichen Ergebnisse erzeugen:

\begin{table}[h]
	\centering
		\begin{center}
			\begin{tabular}{ l | c c }
				\toprule
				Achse & Ergebnisknoten ID's & Ergebnisgröße\\
				\midrule
				descendants & ... & ... \\  
				\bottomrule
			\end{tabular}
			\end{center}
	\caption{Ergebnisse der XPath-Berechnung des XPath-Accelerators aus Phase 3 mit nur einer Achse}
	\label{tab:ErgebnisseDerXPathBerechnug}
\end{table}

Da sich die Knoten IDs unterscheiden, zeigen Sie für einige Knoten die jeweilige Identität.

\subsection*{Zeitmamagement}

Benötigte Zeit pro Person (nur Phase 2): \textbf{XXX}


\end{document}
