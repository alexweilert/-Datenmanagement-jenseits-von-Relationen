\documentclass[11pt]{scrartcl}

\usepackage[top=1.5cm]{geometry}
\usepackage{url}
\usepackage{float}
\usepackage{listings}
\usepackage{xcolor}
\usepackage{graphicx}
\usepackage{booktabs}

\setlength{\parindent}{0em}
\setlength{\parskip}{0.5em}

\newcommand{\youranswerhere}{[Your answer goes here \ldots]}
\renewcommand{\thesubsection}{\arabic{subsection}}

\lstdefinestyle{dmrsql}{
  language=SQL,
  basicstyle=\small\ttfamily,
  keywordstyle=\color{magenta!75!black},
  stringstyle=\color{green!50!black},
  showspaces=false,
  showstringspaces=false,
  commentstyle=\color{gray}}

\lstdefinestyle{dmrJava}{
  language=JAVA,
  basicstyle=\small\ttfamily,
  keywordstyle=\color{magenta!75!black},
  stringstyle=\color{green!50!black},
  showspaces=false,
  showstringspaces=false,
  commentstyle=\color{gray}}

\title{
  \textbf{\large Projektaufgabe 3 } \\
  Phase 2 – Implementierung des XPath Accelerators (5 P) \\
  {\large Datenmanagement jenseits von Relationen}
}

\author{
	Gruppen Nummer 8 \\
	\large Weilert Alexander, 12119653 \\
	\large Jovanovic Dragana, 11850325
}

\begin{document}

\maketitle\thispagestyle{empty}

Dieses Reporting Template dient der Vorbereitung der Abgabe von Phase 2.

\subsection*{Verwendung der Daten aus der DBLP (1 Punkt)}

Geben Sie die folgende Kennzahl bezogen auf \texttt{my\_small\_bib.xml} an:

\begin{itemize}
	\item Anzahl Publikationen von 'Nikolaus Augsten'\footnote{Korrektheit und Vollständigkeit können Sie unter https://dblp.org/pid/76/3961.html überprüfen.}
		\begin{itemize}
		\item ICDE: 8
		\item VLDB: 6
		\item SIGMOD: 5
	\end{itemize}
	\item Position (Zeilennummer von, bis) der Publikationen des Toy-Beispiels in \texttt{my\_small\_bib.xml}
	
	\begin{itemize}
		\item SchmittKAMM23: 287697 $-$ 287712 Lines
		\item HutterAK0L22: 23718 $-$ 23731 Lines
		\item SchalerHS23: 279305 $-$ 279318 Lines
	\end{itemize}
	\item Größe der Datei \texttt{my\_small\_bib.xml} in kB und Zeilenanzahl: 13.065 KB \& 337.405 Zeilen
\end{itemize}



\subsection*{Schemaerstellung (1 Punkt).}
Geben Sie die Create Table statements aller von Ihnen angelegten Tabellen an:
\begin{lstlisting}[style=dmrsql]
CREATE TABLE IF NOT EXISTS node (id int ,s_id TEXT, type TEXT,
		content TEXT)
CREATE TABLE IF NOT EXISTS edge (parents INT, childs INT)

CREATE TABLE IF NOT EXISTS accel (pre INT, post INT, parent INT,
		kind VARCHAR(255), name VARCHAR(255))

CREATE TABLE IF NOT EXISTS content (pre INT, text TEXT)

CREATE TABLE IF NOT EXISTS attribute (pre INT, text TEXT)
\end{lstlisting}

\subsection*{Pre-/Post-Order Annotation (1 Punkt).}
Definieren Sie sich einen \textit{Ausschnitt} aus \texttt{toy\_example.txt}.
Berechnen Sie hierfür händisch die Pre-/Post-Order Annotation.
Inkludieren Sie diese als Foto / Screenshot oder ähnliches in dieses Dokument.
Zeigen Sie dann, dass Ihre Lösung hierfür dieselben Ergebnisse liefert.
Sie können Die Mittels Foto / Screenshot belegen oder in der Übung zeigen.

\subsection*{Achse als Fenster (2 Punkt)}
Zeigen Sie, dass die Berechnung der Achsen für die Beispiele aus Phase 1 hier die gleichen Ergebnisse erzeugen:

\begin{table}[h]
	\centering
		\begin{center}
			\begin{tabular}{ l | c c }
				\toprule
				Achse & Ergebnisknoten ID's & Ergebnisgröße\\
				\midrule
				ancestor & ... & ... \\ 
				descendants & ... & ... \\  
				following SchmittKAMM23& ... & ... \\  
				preceeding SchmittKAMM23 & ... & ... \\ 
				following SchalerHS23& ... & ... \\  
				preceeding SchalerHS23& ... & ... \\ 
				\bottomrule
			\end{tabular}
			\end{center}
	\caption{Ergebnisse der XPath-Berechnung auf Edge Modell aus Phase 1}
	\label{tab:ErgebnisseDerXPathBerechnug}
\end{table}

\begin{table}[h]
	\centering
		\begin{center}
			\begin{tabular}{ l | c c }
				\toprule
				Achse & Ergebnisknoten ID's & Ergebnisgröße\\
				\midrule
				ancestor & ... & ... \\ 
				descendants & ... & ... \\  
				following SchmittKAMM23& ... & ... \\  
				preceeding SchmittKAMM23 & ... & ... \\ 
				following SchalerHS23& ... & ... \\  
				preceeding SchalerHS23& ... & ... \\ 
				\bottomrule
			\end{tabular}
			\end{center}
	\caption{Ergebnisse der XPath-Berechnung des XPath-Accelerators aus Phase 2}
	\label{tab:ErgebnisseDerXPathBerechnug1}
\end{table}

\subsection*{Zeitmamagement}

Benötigte Zeit pro Person (nur Phase 2):
\textbf{Alexander Weilert: 10h} \\
\textbf{Dragana Jovanovic: 9h}


\end{document}
