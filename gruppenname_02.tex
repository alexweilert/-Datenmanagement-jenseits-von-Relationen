\documentclass[11pt]{scrartcl}

\usepackage[top=1.5cm]{geometry}
\usepackage{url}
\usepackage{float}
\usepackage{listings}
\usepackage{xcolor}

\setlength{\parindent}{0em}
\setlength{\parskip}{0.5em}

\newcommand{\youranswerhere}{[Your answer goes here \ldots]}
\renewcommand{\thesubsection}{\arabic{subsection}}

\lstdefinestyle{dmrsql}{
  language=SQL,
  basicstyle=\small\ttfamily,
  keywordstyle=\color{magenta!75!black},
  stringstyle=\color{green!50!black},
  showspaces=false,
  showstringspaces=false,
  commentstyle=\color{gray}}

\lstdefinestyle{dmrJava}{
  language=JAVA,
  basicstyle=\small\ttfamily,
  keywordstyle=\color{magenta!75!black},
  stringstyle=\color{green!50!black},
  showspaces=false,
  showstringspaces=false,
  commentstyle=\color{gray}}

\title{
  \textbf{\large Projektaufgabe 2 } \\
  Phase 1 – Legen der Basis (2.5 P) \\
  {\large Datenmanagement jenseits von Relationen}
}

\author{
  Gruppen Nummer (e.g. A1, B5, B3) \\
  \large Lastname1 Firstname1, StudentID1 \\
  \large Lastname2 Firstname2, StudentID2 
}

\begin{document}

\maketitle\thispagestyle{empty}

Dieses Reporting Template dient der Vorbereitung der Abgabe von Phase 1.

\subsection*{Datengenerator für Matrizen mit Sparsity (0.5 Punkte)}

Zeigen Sie den Code der Funktion generate() als Listing oder Screenshot. Gehen Sie (kurz) auf die wesentlichen Aspekte ein. 

\youranswerhere{}

\begin{lstlisting}[style=dmrJava]
[Code der generate Fuktion...]
\end{lstlisting}

\subsection*{Import der Matrizen in das DBMS (0.5 Punkte)}

Geben Sie das Create Table Statement für Matrix $A$ an.

\begin{lstlisting}[style=dmrsql]
[Create Table A ...]
\end{lstlisting}

Für die Übung bereiten Sie eine Demo des Datenimports vor.


\youranswerhere{}

\subsection*{Wahl des Toy Beispiels (0.5 Punkte)}

Geben Sie Matrix $A$ und $B$ als 2D Array an.

\youranswerhere{}

Zeigen Sie die äquivalente Darstellung der Matrix $A$ und $B$ in der Datenbank.

\youranswerhere{}

\subsubsection*{Implementierung von Ansatz 0 (0.5 Punkte)}

Zeigen Sie den Code der Matrixmultiplikation als Listing oder Screenshot. Erläutern Sie (kurz), welche Laufzeit ihr Algorithmus hat und warum das Kriterium keinen Algorithmus mit sub-kubischer Laufzeit zu wählen erfüllt ist. 


\begin{lstlisting}[style=dmrJava]
[Matrixmultiplikation ...]
\end{lstlisting}


\subsubsection*{Implementierung von Ansatz 1 (0.5 Punkte)}
Berechnen Sie von Hand das Ergebnis $C = A \times B$ für ihr Toy Beispiel und geben Sie es nachfolgend an.

\youranswerhere{}

Zeigen Sie, dass Ihr System $C$ korrekt berechnet (z.B. als Screenshot)

\subsection*{Zeitmamagement}

Benötigte Zeit pro Person (nur Phase 1): \textbf{XXX}

\subsection*{References}

\begin{table}[H]
  \centering
  \begin{tabular}{c}
    \hline
    \textbf{Important:} Reference your information sources! \tabularnewline
    Remove this section if you use footnotes to reference your information sources. \tabularnewline
    \hline
  \end{tabular}
\end{table}

\end{document}
