\documentclass[11pt]{scrartcl}

\usepackage[top=1.5cm]{geometry}
\usepackage{url}
\usepackage{float}
\usepackage{listings}
\usepackage{xcolor}
\usepackage{graphicx}

\setlength{\parindent}{0em}
\setlength{\parskip}{0.5em}

\newcommand{\youranswerhere}{[Your answer goes here \ldots]}
\renewcommand{\thesubsection}{\arabic{subsection}}

\lstdefinestyle{dmrsql}{
  language=SQL,
  basicstyle=\small\ttfamily,
  keywordstyle=\color{magenta!75!black},
  stringstyle=\color{green!50!black},
  showspaces=false,
  showstringspaces=false,
  commentstyle=\color{gray}}

\lstdefinestyle{dmrJava}{
  language=JAVA,
  basicstyle=\small\ttfamily,
  keywordstyle=\color{magenta!75!black},
  stringstyle=\color{green!50!black},
  showspaces=false,
  showstringspaces=false,
  commentstyle=\color{gray}}

\title{
  \textbf{\large Projektaufgabe 2 } \\
  Phase 1 – Legen der Basis (2.5 P) \\
  {\large Datenmanagement jenseits von Relationen}
}

\author{
  Gruppen Nummer (e.g. A1, B5, B3) \\
  \large Weilert Alexander, 12119653 \\
  \large Jovanovic Dragana, StudentID2
}

\begin{document}

\maketitle\thispagestyle{empty}

Dieses Reporting Template dient der Vorbereitung der Abgabe von Phase 1.

\subsection*{Datengenerator für Matrizen mit Sparsity (0.5 Punkte)}

Zeigen Sie den Code der Funktion generate() als Listing oder Screenshot.
Gehen Sie (kurz) auf die wesentlichen Aspekte ein.

\youranswerhere{}

\begin{lstlisting}[style=dmrJava]
  public void generate(int l, double sparsity) {
    try (Statement statement = this.connection.createStatement()) {
        statement.execute("DROP VIEW IF EXISTS C");
        statement.execute("DROP TABLE IF EXISTS A, B");
        // Create Table
        statement.execute("CREATE TABLE A (i int, j int, val int)");
        statement.execute("CREATE TABLE B (i int, j int, val int)");

        int[][] matrixA = generateMatrixA(l, sparsity);
        int[][] matrixB = generateMatrixB(l, sparsity);
        insertMatrix("A", matrixA);
        insertMatrix("B", matrixB);

        ansatz0(matrixA, matrixB); // Matrix Calculator per Algorithm
        ansatz1();                 // Matrix Calculator per Select

    } catch (SQLException e) {
        throw new RuntimeException(e);
    }
}

public int[][] generateMatrixA(int l, double sparsity) {
    Random random = new Random();
    int[][] matrixA = new int[l-1][l];
    System.out.println("--- Matrix A ---");
    for (int i = 0; i < ( l - 1 ); i++) {
        for (int j = 0; j < l; j++) {
            if (random.nextDouble() > sparsity) {
                matrixA[i][j] = random.nextInt(1, 11); // Random value between 0 and 10
            } else {
                matrixA[i][j] = 0;
            }
            System.out.print(matrixA[i][j] + " ");
        }
        System.out.println();
    }
    return matrixA;
}

public int[][] generateMatrixB(int l, double sparsity) {
    Random random = new Random();
    int[][] matrixB = new int[l][l-1];
    System.out.println("--- Matrix B ---");
    for (int i = 0; i < l; i++) {
        for (int j = 0; j < (l - 1); j++) {
            if (random.nextDouble() > sparsity) {
                matrixB[i][j] = random.nextInt(1, 11);// Random value between 0 and 10
            } else {
                matrixB[i][j] = 0;
            }
            System.out.print(matrixB[i][j] + " ");
        }
        System.out.println();
    }
    return matrixB;
}

public void insertMatrix(String tableName, int[][] matrix) {
    try (Statement statement = this.connection.createStatement()) {
        StringBuilder insertQuery = new StringBuilder("INSERT INTO " + tableName + " VALUES ");
        for (int i = 0; i < matrix.length; i++) {
            for (int j = 0; j < matrix[i].length; j++) {
                if (matrix[i][j] != 0) {
                    insertQuery.append("(").append(i+1).append(",").append(j+1).append(",").append(matrix[i][j]).append("),");
                }
            }
        }
        insertQuery.deleteCharAt(insertQuery.length() - 1);
        statement.executeUpdate(insertQuery.toString());
    } catch (SQLException e) {
        throw new RuntimeException(e);
    }
}
\end{lstlisting}

\subsection*{Import der Matrizen in das DBMS (0.5 Punkte)}

Geben Sie das Create Table Statement für Matrix $A$ an.

\begin{lstlisting}[style=dmrsql]
  CREATE TABLE A (i int, j int, val int)
\end{lstlisting}

Für die Übung bereiten Sie eine Demo des Datenimports vor.


\youranswerhere{}

\subsection*{Wahl des Toy Beispiels (0.5 Punkte)}

Geben Sie Matrix $A$ und $B$ als 2D Array an.

\includegraphics{2D_Darstellung.jpg}

Zeigen Sie die äquivalente Darstellung der Matrix $A$ und $B$ in der Datenbank.

\includegraphics{table_a.jpg}
\includegraphics{table_b.jpg}

\subsubsection*{Implementierung von Ansatz 0 (0.5 Punkte)}

Zeigen Sie den Code der Matrixmultiplikation als Listing oder Screenshot. Erläutern Sie (kurz), welche Laufzeit ihr Algorithmus hat und warum das Kriterium keinen Algorithmus mit sub-kubischer Laufzeit zu wählen erfüllt ist. 


\begin{lstlisting}[style=dmrJava]
  public void ansatz0(int[][] matrixA, int[][] matrixB) {
    try (Statement statement = this.connection.createStatement()) {
        int[][] result = new int[matrixA.length][matrixB[0].length];
        for (int i = 0; i < matrixA.length; i++) {
            for (int j = 0; j < matrixB[0].length; j++) {
                for (int k = 0; k < matrixA[0].length; k++) {
                    result[i][j] += matrixA[i][k] * matrixB[k][j];
                }
            }
        }
        System.out.println("--- Matrix Calculator ---");
        for (int i = 0; i < result.length; i++) {
            for (int j = 0; j < result.length; j++) {
                System.out.print(result[i][j] + " ");
            }
            System.out.println();
        }
    } catch (SQLException e) {
        throw new RuntimeException(e);
    }
}
\end{lstlisting}


\subsubsection*{Implementierung von Ansatz 1 (0.5 Punkte)}
Berechnen Sie von Hand das Ergebnis $C = A \times B$ für ihr Toy Beispiel und geben Sie es nachfolgend an.

\begin{lstlisting}[style=dmrJava]
public void ansatz1() {
  try (Statement statement = this.connection.createStatement()) {
      statement.execute("CREATE VIEW C AS " +
          "SELECT a.i, b.j, SUM(A.val * B.val) " +
              "FROM a,  b " +
              "WHERE a.j = b.i " +
              "GROUP BY a.i, b.j");

      } catch (SQLException e) {
      throw new RuntimeException(e);
  }
}
\end{lstlisting}

Zeigen Sie, dass Ihr System $C$ korrekt berechnet (z.B. als Screenshot)

\subsection*{Zeitmamagement}

Benötigte Zeit pro Person (nur Phase 1): \\ \\
\textbf{Alexander Weilert: 5h} \\
\textbf{Dragana Jovanovic: 5h}

\subsection*{References}

\begin{table}[H]
  \centering
  \begin{tabular}{c}
    \hline
    \textbf{Important:} Reference your information sources! \tabularnewline
    Remove this section if you use footnotes to reference your information sources. \tabularnewline
    \hline
  \end{tabular}
\end{table}

\end{document}
